\documentclass{article}
\usepackage{graphicx}
\usepackage{amssymb}
\usepackage{amsmath}
\usepackage{mathtools}
\usepackage{tensor}
\usepackage{bm}

\title{Drag Projectile}
\author{Aidan Sgarlato}
\date{February 2023}
\begin{document}

\maketitle

\section{Theoretical Derivation}
\noindent
Quadratic and Linear Drag equations of motion

$$
my''=-mg-c_{1}y'-c_{2}y'\sqrt{y'^{2}+x'^{2}}
$$
$$
mx''=-c_{1}x'-c_{2}x'\sqrt{y'^{2}+x'^2}
$$
By analyzing the coefficients it's clear that quadratic drag will dominate in juggling conditions. These conditions will also have high initial launch angle and as such the velocity in the x direction is small and the velocity in the y is large. So we will use a first order Taylor approximation around zero of the $x'$ function and a first order asymptotic expansion around $y'$ for each of the respective equations. These two expansions generally look as such...
\\
\\
Taylor series around $z=0$:
$$
z\sqrt{z^{2}+a^{2}} = z|a|+O(z^{3})
$$
Asymptotic Expansion:
$$
z\sqrt{z^{2}+a^{2}} = z|z|+sgn(z)\frac{a^{2}}{2} + O(\frac{1}{z^{2}})
$$
The problem with the asymtotic approximations is that they are discontinuous when $y'=0$ and the asymptotic expansion is also not a very good approximation near $y'=0$. This all occurs at the apex of the the projectile motion. The assumption that the drag is quadratic also fails at the apex as the total velocity is so low. Hence, we have three domains for the differential equation: before the apex, near the apex, and after the apex. The solutions of which can be stitched together to be made to be $C^{1}$ continuous.
\\
\\
Regime 1:
$$
0 \ll y'
$$
$$
my''=-mg-c_{1}y'-c_{2}y'^{2}-\frac{c_{2}}{2}x'^{2}
$$
$$
mx''=-c_{1}x'-c_{2}y'x'
$$
Regime 2:
$$
|y'| < \epsilon
$$
$$
my''=-mg-c_{1}y'
$$
$$
mx''=-c_{1}x'
$$
Regime 3:
$$
y' \ll 0
$$
$$
my''=-mg-c_{1}y'+c_{2}y'^{2}+\frac{c_{2}}{2}x'^{2}
$$
$$
mx''=-c_{1}x'+c_{2}y'x'
$$
The solution for $x'$ in all regimes is the same up to a different drag coefficient and integrating constant.
$$
x'=\chi e^{-\frac{-c_{1}}{m}t \pm \frac{c_{2}}{m}y}
$$
Within this equation we will assume that $y$ is of a linear nature. This approximation can be justified because this solution for $x'$ is only being used in the high velocity regimes of $y'$ where, with our intuition of the vacuum solution of the projectile, such a linear approximation should be closest to what it is approximating. This approximation looks as such...
$$
x' = \chi e^{(-\frac{c_{1}}{m} \pm \frac{c_{2}a_{\pm}}{m})t} \quad , \quad y=a_{\pm}t
$$
Where $a_{-}$ corresponds to regime 1 and $a_{+}$ to regime 3. Therefore the $y$ equations of motion are as such...
\\
\\
Regime 1:
$$
y''=-g-\frac{c_{1}}{m}y'-\frac{c_{2}}{m}y'^{2}-\frac{c_{2}\chi^{2}_{1}}{2m}e^{-2(\frac{c_{1}}{m} + \frac{c_{2}a_{-}}{m})t}
$$
Regime 3:
$$
y''=-g-\frac{c_{1}}{m}y'+\frac{c_{2}}{m}y'^{2}+\frac{c_{2}\chi^{2}_{3}}{2m}e^{-2(\frac{c_{1}}{m} - \frac{c_{2}a_{+}}{m})t}
$$
Both these equations are Riccati type differential equations and as such they can be transformed as such. We will use $u$ for regime 1 and $v$ for regime 3.
\\
\\
Regime 1:
$$
u''+\frac{c_{1}}{m}u'+(\frac{gc_{2}}{m}+\frac{c_{2}^{2}\chi^{2}_{1}}{2m^{2}}e^{-\frac{2}{m}(c_{1}+c_{2}a_{-})t})u=0 \quad , \; y= \frac{m}{c_{2}}ln(u)
$$
Regime 3:
$$
v''+\frac{c_{1}}{m}v' + (-\frac{gc_{2}}{m}+\frac{c_{2}^{2}\chi^{2}_{3}}{2m^{2}}e^{-\frac{2}{m}(c_{1}-c_{2}a_{+})t})v=0 \quad , \; y= -\frac{m}{c_{2}}ln(v)
$$
Both of these equations are of a similar form and can be transformed very similarly.
$$
u(t) = x^{\frac{c_{1}}{2(c_{1}+c_{2}a_{-})}}\mu(x) \quad , \quad x = \frac{c_{2}\chi_{1}}{\sqrt{2}(c_{1}+c_{2}a_{-})}e^{-\frac{c_{1}+c_{2}a_{-}}{m}t}
$$
$$
v(t) = x^{\frac{c_{1}}{2(c_{1}-c_{2}a_{+})}}\nu(x) \quad , \quad x = \frac{c_{2}\chi_{1}}{\sqrt{2}(c_{1}-c_{2}a_{+})}e^{-\frac{c_{1}-c_{2}a_{+}}{m}t}
$$
This change in coordinate system and substitution yields two differential equations of the form...
$$
x^{2}\mu''+x\mu'+(x^2+\zeta^{2})\mu=0
$$
$$
x^{2}\nu''+x\nu'+(x^2-\Omega^{2})\nu=0
$$
Where...
$$
\zeta  = \frac{\sqrt{c_{2}gm-\frac{c_{1}^{2}}{4}}}{c_{1}+c_{2}a_{-}}
$$
$$
\Omega =  \frac{\sqrt{c_{2}gm+\frac{c_{1}^{2}}{4}}}{c_{1}-c_{2}a_{+}}
$$
These are the canonical differential equations that define the Bessel function of the first kind.
$$
\mu = k_{1}J_{i\zeta}(x)+k_{2}J_{-i\zeta}(x)
$$
$$
\nu = l_{1}J_{\Omega}(x)+l_{2}J_{-\Omega}(x)
$$

$$
J_{\alpha}(t) = \sum^{\infty}_{n=0} \frac{(-1)^{n}}{n!\Gamma (n+\alpha+1)}\bigg(\frac{t}{2}\bigg)^{2n+\alpha}
$$
We can then use these results to find the form of the y equations...
\\
\\
Regime 1:
$$
y=-\frac{c_{1}}{2 c_{2}}t+\frac{m}{c_{2}}ln\bigg( k_{1}J_{i\zeta}\bigg( \frac{c_{2}\chi_{1}}{\sqrt{2}(c_{1}+c_{2}a_{-})} e^{-\frac{c_{1}+c_{2}a_{-}}{m}t} \bigg)+k_{2}J_{-i\zeta}\bigg( \frac{c_{2}\chi_{1}}{\sqrt{2}(c_{1}+c_{2}a_{-})} e^{-\frac{c_{1}+c_{2}a_{-}}{m}t} \bigg) \bigg)
$$
Regime 3:
$$
y=\frac{c_{1}}{2 c_{2}}t-\frac{m}{c_{2}}ln\bigg(l_{1}J_{\Omega}\bigg( \frac{c_{2}\chi_{3}}{\sqrt{2}(c_{1}-c_{2}a_{+})} e^{-\frac{c_{1}-c_{2}a_{+}}{m}t} \bigg)+l_{2}J_{-\Omega}\bigg( \frac{c_{2}\chi_{3}}{\sqrt{2}(c_{1}-c_{2}a_{+})} e^{-\frac{c_{1}-c_{2}a_{+}}{m}t} \bigg)\bigg)
$$
We can now begin the stitching of these solutions together. We can first determine $\chi_{1}$ from the $x$ solution.
$$
x'(0) = cos(\theta)v_{0} = \chi_{1}
$$
When looking at the free parameters of the equation in regime 1 it is clear to see that the two parameters are complex conjugates of one another as the argument of the logarithm must be real and the Bessel functions are complex conjugates of one another because the Bessel function is holomorphic and real to real in the argument of it's order. As such we will say $k_{1}=\frac{\alpha +i\beta}{2}$ and $k_{2}=\frac{\alpha -i\beta}{2}$ and y in regime 1 can then be written..
$$
y = -\frac{c_{1}}{2c_{2}}t+\frac{m}{c_{2}}ln \bigg( \alpha \Re \bigg( J_{i\zeta}\bigg( \xi e^{-\frac{c_{1}+c_{2}a_{-}}{m}t} \bigg) \bigg) -\beta \Im \bigg( J_{i\zeta}\bigg( \xi e^{-\frac{c_{1}+c_{2}a_{-}}{m}t} \bigg) \bigg)\bigg)
$$
$$
\xi = \frac{c_{2}v_{0}cos(\theta)}{\sqrt{2}(c_{1}+c_{2}a_{m})}
$$
To simplify the the proceeding equation we will adopt the notation below.
$$
J_{i\zeta}(x) = \mathfrak{M}_{\zeta}(x)+i\mathfrak{N}_{\zeta}(x)
$$
$$
J_{i\zeta+1}(x)-J_{i\zeta-1}(x) = \mathfrak{F}_{\zeta}(x)+i\mathfrak{G}_{\zeta}(x)
$$
$$
\mathfrak{M}'_{\zeta}(x) = -\frac{1}{2}\mathfrak{F}_{\zeta}(x) \quad , \quad \mathfrak{N}'_{\zeta}(x) = -\frac{1}{2}\mathfrak{G}_{\zeta}(x)
$$
Using thi
s notation above we can solve for these free parameters using the initial conditions...
$$
\alpha = \frac{1}{\Xi}\bigg(\mathfrak{N}_{\zeta}(\xi)\bigg(2\sqrt{2}tan(\theta)+ \frac{c_{1}\sqrt{2}}{v_{0}c_{2}}sec(\theta)\bigg)-\mathfrak{G}_{\zeta}(\xi) \bigg)
$$
$$
\beta =  \frac{1}{\Xi}\bigg(\mathfrak{M}_{\zeta}(\xi)\bigg(2\sqrt{2}tan(\theta)+ \frac{c_{1}\sqrt{2}}{v_{0}c_{2}}sec(\theta)\bigg)-\mathfrak{F}_{\zeta}(\xi) \bigg)
$$
$$
\Xi = \mathfrak{N}_{\zeta}(\xi)\mathfrak{F}_{\zeta}(\xi)-\mathfrak{M}_{\zeta}(\xi)\mathfrak{G}_{\zeta}(\xi)
$$
Using this notation we can reformulate the equations of motion in regime 1 as such..
$$
x_{1}(t) = \frac{mv_{0}cos(\theta)}{c_{1}+c_{2}a_{-}}\bigg( 1-e^{-\frac{c_{1}+c_{2}a_{-}}{m}t} \bigg)
$$
$$
y_{1}(t) = -\frac{c_{1}}{2 c_{2}}t +\frac{m}{c_{2}}ln\bigg(\alpha \mathfrak{M}_{\zeta}(\xi e^{-\frac{c_{1}+c_{2}a_{-}}{m}t})-\beta \mathfrak{N}_{\zeta}(\xi e^{-\frac{c_{1}+c_{2}a_{-}}{m}t})\bigg)
$$
The specific forms of these functions at this point is inconsequential as each have a wide variety of series and integral representations. The specific nature of the purely imaginary order or complex valued orders with arguments very near $\pm \pi/2$ as we have here is under studied. Many representations are not applicable for such values of the order argument. More detail will be put into this topic for when we try to effectively implement the solution into computer code. We can now do work to stitch this solution with the solution of regime two. The second regime is trivially solvable as such.
$$
x_{2}(t) = -\delta_{1}\frac{m}{c_{1}}e^{-\frac{c_{1}}{m}t}+\delta_{2}
$$
$$
y_{2}(t) = -\lambda_{1} \frac{m}{c_{1}}e^{-\frac{c_{1}}{m}t}-\frac{gm}{c_{1}}t+\lambda_{2}
$$
The solution to these parameters looks as such.
$$
\delta_{1} = v_{0}cos(\theta)e^{-\frac{c_{2}a_{-}}{m}\tau_{1}}
$$
$$
\delta_{2} = mv_{0}cos(\theta)\bigg(\frac{1}{c_{1}+c_{2}a_{-}}+e^{-\frac{c_{1}+c_{2}a_{-}}{m}\tau_{1}}\bigg(\frac{1}{c_{1}}-\frac{1}{c_{1}+c_{2}a_{-}}\bigg)\bigg)
$$
$$
\lambda_{1} = \bigg(\frac{gm}{c_{1}}-\frac{c_{1}}{2c_{2}} \bigg)e^{\frac{c_{1}}{m}\tau_{1}}+\frac{v_{0}cos(\theta)}{2\sqrt{2}}e^{-\frac{a_{-}c_{2}}{m}\tau_{1}}\frac{\alpha \mathfrak{F}_{\zeta}(\xi')-\beta\mathfrak{G}_{\zeta}(\xi')}{\alpha \mathfrak{M}_{\zeta}(\xi')-\beta\mathfrak{N}_{\zeta}(\xi')}
$$
$$
\lambda_{2} = \bigg( \frac{gm}{c_{1}}-\frac{c_{1}}{2c_{2}} \bigg)(\tau_{1}+\frac{m}{c_{1}})+\frac{mv_{0}cos(\theta)}{2\sqrt{2}c_{1}}e^{-\frac{c_{1}+c_{2}a_{-}}{m}\tau_{1}}\frac{\alpha \mathfrak{F}_{\zeta}(\xi')-\beta\mathfrak{G}_{\zeta}(\xi')}{\alpha \mathfrak{M}_{\zeta}(\xi')-\beta\mathfrak{N}_{\zeta}(\xi')}+\cdots 
$$
$$
\cdots + \frac{m}{c_{2}}ln\bigg( \alpha\mathfrak{M}_{\zeta}(\xi')-\beta\mathfrak{N}_{\zeta}(\xi') \bigg)
$$
$$
\xi'= \xi e^{-\frac{c_{1}+c_{2}a_{-}}{m}\tau_{1}}
$$
We then can move onto the third regime and stitch it together with our solution from the second at some point $\tau_{2}$ which we know to be greater than $tau_{1}$ and will be just after the time of the apex of the projectile. The equations of motion for the third regime are as follows.
$$
x_{3}(t) = -\frac{m\chi_{3}}{c_{1}-c_{2}a_{+}}e^{-\frac{c_{1}-c_{2}a_{+}}{m}t}+d
$$
$$
y_{3}(t)=\frac{c_{1}}{2 c_{2}}t-\frac{m}{c_{2}}ln\bigg(l_{1}J_{\Omega}\bigg( \frac{c_{2}\chi_{3}}{\sqrt{2}(c_{1}-c_{2}a_{+})} e^{-\frac{c_{1}-c_{2}a_{+}}{m}t} \bigg)+l_{2}J_{-\Omega}\bigg( \frac{c_{2}\chi_{3}}{\sqrt{2}(c_{1}-c_{2}a_{+})} e^{-\frac{c_{1}-c_{2}a_{+}}{m}t} \bigg)\bigg)
$$
Solving for $\chi_{3}$ and $d$ give. 
$$
\chi_{3} = v_{0}cos(\theta)e^{-\frac{c_{2}}{m}(a_{-}\tau_{1}+a_{+}\tau_{2})}
$$
$$
d = mv_{0}cos(\theta) \bigg(\frac{1}{c_{1}+c_{2}a_{-}} + e^{-\frac{c_{2}a_{-}\tau_{1}+c_{1}\tau_{2}}{m}}\bigg( \frac{1}{c_{1}-c_{2}a_{+}}-\frac{1}{c_{1}} \bigg)+e^{-\frac{c_{1}+c_{2}a_{-}}{m}\tau_{1}}\bigg(\frac{1}{c_{1}}-\frac{1}{c_{1}+c_{2}a_{-}}\bigg) \bigg)
$$
Taking these values we can give an expression for $y$.
$$
y_{3}(t)=\frac{c_{1}}{2 c_{2}}t-\frac{m}{c_{2}}ln\bigg(l_{1}J_{\Omega}\bigg( \psi e^{-\frac{c_{1}-c_{2}a_{+}}{m}t} \bigg)+l_{2}J_{-\Omega}\bigg( \psi e^{-\frac{c_{1}-c_{2}a_{+}}{m}t} \bigg)\bigg)
$$
$$
\psi = \frac{c_{2}v_{0}cos(\theta)}{\sqrt{2}(c_{1}-c_{2}a_{+})}e^{-\frac{c_{2}}{m}(a_{-}\tau_{1}+a_{+}\tau_{2})}
$$
These generate a system of equation we can use to solve for the remaining free parameters.
$$
l_{1}J_{\Omega}(\psi')+l_{2}J_{-\Omega}(\psi) = e^{\Lambda}
$$
$$
l_{1}\bigg( J_{\Omega+1}(\psi')-J_{\Omega-1}(\psi') \bigg)+l_{2}\bigg( J_{-\Omega+1}(\psi')-J_{-\Omega-1}(\psi') \bigg) = \bigg( \frac{c_{1}}{2c_{2}}+\frac{gm}{c_{1}}-\lambda_{1} e^{-\frac{c_{1}}{m}\tau_{2}} \bigg) \frac{e^{\Lambda} 2\sqrt{2}}{v_{0}cos(\theta)}e^{\frac{c_{2}}{m}(a_{-}\tau_{1}+a_{+}\tau_{2})}
$$
$$
\Lambda =  \lambda_{1}\frac{c_{2}}{c_{1}}e^{-\frac{c_{1}}{m}\tau_{2}}+\bigg( \frac{gc_{2}}{c_{1}} + \frac{c_{1}}{2m}\bigg)\tau_{2}-\frac{c_{2}\lambda_{2}}{m}
$$
$$
\psi' = \frac{c_{2}v_{0}cos(\theta)}{\sqrt{2}(c_{1}-c_{2}a_{+})}e^{-\frac{c_{1}\tau_{2}+c_{2}a_{-}\tau_{1}}{m}}
$$
Solving this system we find the equations of motion in regime 3.
$$
x_{3}(t) = -\frac{\sqrt{2}m\psi}{c_{2}}e^{-\frac{c_{1}-c_{2}a_{+}}{m}t} + d
$$
$$
y_{3}(t) = \frac{c_{1}}{2 c_{2}}t-\frac{m}{c_{2}}ln \bigg(\mu J_{\Omega}\bigg( \psi e^{-\frac{c_{1}-c_{2}a_{+}}{m}t} \bigg) - \nu J_{-\Omega}\bigg( \psi e^{-\frac{c_{1}-c_{2}a_{+}}{m}t} \bigg) \bigg) + \frac{m}{c_{2}}\bigg(ln ( \Psi ) - \Lambda \bigg)
$$
$$
\mu = J_{-\Omega}(\psi')\bigg( \frac{c_{1}}{2c_{2}}+\frac{gm}{c_{1}}-\lambda_{1}e^{-\frac{c_{1}}{m}\tau_{2}} \bigg)\frac{2c_{2}}{(c_{1}-c_{2}a_{+})\psi}-J_{-\Omega+1}(\psi')+J_{-\Omega-1}(\psi')
$$
$$
\nu = J_{\Omega}(\psi')\bigg( \frac{c_{1}}{2c_{2}}+\frac{gm}{c_{1}}-\lambda_{1}e^{-\frac{c_{1}}{m}\tau_{2}} \bigg)\frac{2c_{2}}{(c_{1}-c_{2}a_{+})\psi}-J_{\Omega+1}(\psi')+J_{\Omega-1}(\psi')
$$
$$
\Psi = J_{-\Omega}(\psi') \bigg( J_{\Omega+1}(\psi')-J_{\Omega-1}(\psi') \bigg) -J_{\Omega}(\psi') \bigg( J_{-\Omega+1}(\psi')-J_{-\Omega-1}(\psi') \bigg)
$$
\end{document}
